\documentclass{article}

\begin{document}

\title{Techno-Realist Anarchism, or Techno-Anarchism}
\author{}
\date{}
\maketitle

\section{Preface}

There are two highly influential and harmful trends in technology politics: techno-pessimism on the left which is fearful or apathetic towards new developments in technology, and techno-utopianism on the right which considers all new technological advancements as inherently trending towards liberty. There is obviously a third option which sees the potential of technology as an arena of political struggle which can swing in favor of one side or another, depending on who wins the struggle. There are a few who have recognized this, but it has not yet risen to the level of a movement. The below points are not presented as definite truth, but as a starting point for discussion. The hope is that others will join the conversation to refine these ideas into a coherent movement.

\section{The need for a techno-anarchist movement}
\begin{enumerate}
\item The computer revolution is a revolution on the scale of the industrial revolution: with social, economic, and political consequences.
\item Political movements of traditional liberals, conservatives, and socialists have been slow to identify how radical this transition is, but this is starting to change now.
\item Pro-capitalism libertarians have been quicker to catch on to the significance of this than left-wing movements. However, the advantage is not so great because they were wrong about most of their assumptions.
\item The traditional left-wing (Marxism, anarchism, and the various other kinds of socialism) has been particularly slow to realize the liberatory potential of technology.
\item Part of the problem is that very many technology workers are middle class professionals, and so their class interests are divided from non-technology workers.
\item Part of the problem is that established pro-capitalism movements associated with technology have actively driven away potential collaborators, as can be seen in spaces like cryptocurrency.
\item In anticipation of the next round of revolutions, anarchists would be wise to survey the territory and understand how their ideology relates to new technological developments.
\item The crypto-anarchist movement is mainly a pro-capitalism movement and therefore incapable of moving to overthrow the old system. A movement of resistance requires a more left-wing analysis.
\item Pro-capitalism movements cannot be allowed to monopolize the conversation regarding the influence of technology on social transformation. The concept is more generally applicable than that.
\item It is necessary to take the good ideas of the crypto-anarchist movement, without falling into the trap of techno-utopianism and pro-capitalism thinking.
\item One of the main issues of the crypto-anarchist movement is that its concept of freedom only serves the technologically literate by providing an escape, and not a general framework to resist power.
\item Another issue is the pervasive idea of meritocracy. Meritocracy will not lead to a more free society, because those with power will always have the ability to game the system. Freedom is made possible when power is made impossible.
\item Left/right syncretic movements are abound in the technological activist space. The result is not a constructive synthesis, but a kind of confused and directionless mess.
\item This political confusion is inevitable because technological activists are working towards cross purposes, and working with completely different assumptions of how the world works.
\item The lack of direction from syncretic movements has translated to a lack of effective action, because activists are working at cross purposes.
\item Instead, it is better that technological activists splinter off into different groups based on ideology and only collaborate with each other where it makes sense.
\item Unity between authoritarian socialists and anarchists is not a viable path forward, and has resulted in a similar political confusion to left/right syncretic movements.
\item Authoritarian leftists are, in truth, afraid of the computer revolution. They see that it has made it more difficult to govern and easier to resist state power, which is dangerous for authoritarians.
\item Authoritarian socialists and anarchists are natural enemies, competing with each other for influence during a revolutionary period. Cooperation is only possible because the current revolution is not recognized as such.
\item There needs to be the intentional development of a techno-anarchist movement which sees itself as different from other anarchist movements, and different from pro-capitalism crypto-anarchist movements.
\end{enumerate}

\section{The goal of techno-anarchism}
\begin{enumerate}
\setcounter{enumi}{20}
\item The problem of modern society is bureaucracy. A bureaucracy is a form of organization based on specialization of labor, hierarchy, formal rules, and impersonal interaction.
\item In its place should be the generalization of labor, horizontal organization, folkways and general principles above formal rules, and interactions based on personal trust. These are anti-bureaucratic principles.
\item Bureaucracy has advantages and disadvantages, but the advantages overwhelmingly belong to the managers and the disadvantages overwhelmingly belong to the managed.
\item Some bureaucracies are not so bad, only because they have not been given enough time to develop. Eventually, every bureaucratic organization loses its original purpose and instead serves only to perpetuate its own existence.
\item Capitalism does not solve the problem either. In fact, corporations follow the exact same principles of bureaucracy as the modern state: specialization, hierarchy, formality, and impersonality.
\item The individual is dehumanized and controlled under capitalism just as under the authoritarian state. That is why individuals are unhappy under capitalism, for reasons unrelated to the authoritarian state itself.
\item Markets are not the same as capitalism. Capitalism is inherently bureaucratic, but a market-based economy can still exist under the anti-bureaucratic principles outlined earlier.
\item Anarcho-capitalism is not actually radical because it defends existing property rights. If anarcho-capitalism were really opposed to the current economic structure, it would need to develop a theory of expropriation.
\item Anti-bureaucratic movements are necessarily anti-capitalism, but not all anti-capitalism movements are anti-bureaucratic. There must be a multi-dimensional struggle against the various bureaucratic structures that dominate society.
\item A mass movement based on class is difficult, and there is not a clear two-class system of worker and capitalist. Highly-paid professionals like software engineers do not obviously have a shared class interest with those who barely make enough income to survive.
\item A mass movement based on bureaucracy is more attainable, because the bureaucratic system is necessarily bottom-heavy without exceptions.
\item Capitalism is the bureaucratic institution that most people interact with on a daily basis, in comparison to the state. However, the state is the more ancient and experienced enemy which perfected bureaucracy first.
\item Nationalism is another dimension of anti-bureaucratic struggle. Nationalism is the means by which the managers of a state identify their own interests with those of the managed, instead of the managed of different states identifying a shared interest against the managers.
\item Nationalism is arbitrary because it is based on a coercive relationship to a state, which is citizenship. Instead of having positive feelings, citizens should feel overwhelming resentment towards the state that rules them.
\item Instead, people should be able to identify with whatever voluntary association that pleases them, or to identify as an individual.
\item In the long view of history, it seems that nationalism as an institution is waning, but it is still a major threat and the trajectory of history may change quickly.
\item Pro-bureaucratic actors are aware that they are losing power in the short-term, and can get the upper hand if anti-bureaucratic actors are not careful.
\item It cannot be assumed that advancements in technology, science, and social developments will automatically translate into greater freedom even if that seems to be the current trend.
\item It is necessary for anti-bureaucratic actors to be aware of new technological developments, and not to idly allow the arc of history to do its business.
\item Instead, anti-bureaucratic actors need to intentionally think about what they want from the world and how to get it.
\end{enumerate}

\section{Techno-anarchism and other anarchisms}
\begin{enumerate}
\setcounter{enumi}{40}
\item Techno-anarchism can be considered similar to anarcho-syndicalism, where the goal and strategy are not completely separated.
\item Alternatively, techno-anarchism can be considered exclusively as a kind of strategy without a particular goal in mind.
\item The best way may be a compromise, which is to take techno-anarchism as an update or modification to all other anarchist schools of thought.
\item Techno-anarchism does not, by itself, offer a complete strategy or a complete outline of the goal. It requires other kinds of anarchism as a basis for its action.
\item Panarchy is the generalization of all anarchisms coexisting as voluntary communities, dynamically negotiating terms with each other.
\item Voluntarily bureaucratic forms of organization will also be possible under panarchy, which is unfortunate but necessary.
\item People generally prefer to be self-managed, so voluntarily bureaucratic forms of organization will gradually transition to anti-bureaucratic forms of organization.
\item Techno-anarchism should tend towards panarchistic thinking, to ensure that its ideas can spread among various kinds of anarchists.
\item Several anarchisms have already developed in response to new changes in technology. Techno-anarchism should consider incorporating these insights.
\item Anarcho-primitivism is the position that technology is essentially a negative force. Techno-anarchism distinguishes between good and bad technology based on its particular characteristics.
\item Anarcho-transhumanism is the position that technological development should be accelerated to expand human potential. Techno-anarchism is mindful that some new technologies inherently favor bureaucracy and should be avoided.
\item Post-scarcity anarchism is the position that technology is making economic organization obsolete. Techno-anarchism is careful not to replace the human managers with machines, while maintaining the old bureaucracy.
\item The difference that techno-anarchism makes is to integrate computers into these visions of the future, just as industrial production was integrated into the earlier anarchist visions of the future.
\item For market anarchists, there is must be a deeper analysis of the radical potential of cryptocurrency and decentralized production through 3D-printing.
\item For anarcho-syndicalists, there must be a deeper analysis of the economic consequences of the computing revolution which have not fully played out yet.
\item For anarcho-communists, there must be a deeper analysis of the potential of computers to enhance democracy, community organization, and decentralized communication.
\item Techno-anarchists need to develop a unique theory of economics, politics, and sociology based on the changes associated with the computer revolution.
\item The expectation is that techno-anarchism will be more systems or process oriented, as some left/right syncretic movements attempted. They key is to maintain a broader anti-bureaucratic framework where this analysis takes place.
\item Despite that, the key insight of techno-anarchism is the development of a strategy of resistance against bureaucracy. That is its contribution to the whole of the anarchist movement.
\item Ultimately, even if a distinct techno-anarchist movement fails to develop, it will still be a success if other anarchists can learn from the insights of techno-anarchism.
\end{enumerate}

\section{Techno-anarchism as a strategy}
\begin{enumerate}
\setcounter{enumi}{60}
\item Technology is not a neutral thing which takes on the intent of its wielder. Technology demands and facilitates particular kinds of social structures.
\item By guiding society towards the adoption of certain kinds of technologies, society can be guided towards the anti-bureaucratic social structure that is desired.
\item It is necessary to attack, subvert, evade, and destroy technologies that are observed to have greater potential for bureaucratic than anti-bureaucratic social structures.
\item In contrast, it is necessary to develop, distribute, advocate for, and support technologies that are observed to have greater potential for anti-bureaucratic than bureaucratic social structures.
\item The essence of techno-anarchism is to utilize technologies which asymmetrically give control to those at the bottom of bureaucracies, and deprive control from those at the top.
\item This is already being done through the work of whistleblowing activists, who have used anonymous drops to distribute state secrets.
\item There is also the possibility of actively using networks to disrupt those systems critical to the functioning of these bureaucracies.
\item Anti-bureaucratic replacements must be built for bureaucratic systems relied upon today. Cryptocurrency has created a substitute for national currencies which is decentralized in its control.
\item Proprietary software is dependent on copyright, which is dependent on state power. There must be free software replacements for all proprietary software.
\item Most people are still on centralized social network services, which are controlled by a few powerful corporations. There must be federated and peer-to-peer replacement for these.
\item The internet has, so far, been more favorable to anti-bureaucratic actors than pro-bureaucratic actors.
\item The full potential of the internet has not been realized yet. Once the danger is properly identified, pro-bureaucratic actors will seek to restrict it.
\item The internet is not as difficult to shut down as some technological activists seem to believe. Online censorship in the People's Republic of China has been hugely successful thus far.
\item The United States and other countries will eventually copy the policies of People's Republic of China in its censorship of the internet.
\item The internet is dependent on small number of powerful corporations and can easily be disabled by nation states. There needs to be a decentralized internet.
\item There are a few specific projects that techno-anarchists need to seriously consider in addition to cryptography-related ones.
\item Hardware production is still centralized and depends on a small number of powerful corporations. There needs to be decentralized production of computing hardware using free blueprints.
\item Certain network services have not been done effectively in a decentralized way, such as search engines. There needs to be advancements in these kinds of services.
\item There is good work on encryption and peer-to-peer networks being done. At this point, the issue is more a matter of usability, awareness, and adoption of these technologies.
\item A large part of the work of techno-anarchists is to educate the public about encryption technology, why it is important, and how to use it.
\end{enumerate}

\section{A techno-anarchist organization}
\begin{enumerate}
\setcounter{enumi}{80}
\item A techno-anarchist organization serves a few purposes: political education, building anti-bureaucratic replacements for bureaucratic institutions, and building up strength to oppose bureaucratic institutions.
\item In addition to educating the general public, techno-anarchist groups should educate non-technical anarchists on the use of these technologies and to create tools and platforms anarchists can use.
\item The natural platform for techno-anarchists to organize is the internet. In particular, the most likely places are encrypted chat rooms, forums, and social media sites.
\item However, virtual interaction cannot be a replacement for physical organizing. There is still a need to maintain physical presence in local communities.
\item The purpose of techno-anarchism should not be virtual secession, but the subversion of bureaucratic power in the real world.
\item There is a particular problem that bureaucracies solve, which is to organize a large number of people who do not have a trusted relationship with each other.
\item Anarchists have recognized two anti-bureaucratic alternatives, which are direct democracy and markets. Many denounce one and accept the other, but it is better to take both as options.
\item Consensus-based organization is different because it requires a relationship of trust, and therefore is not solving the same problem. Bad faith actors are too powerful under consensus.
\item Federalism is also different, being one system on top of another system. It offers helpful compromises, such as a directly democratic federation of consensus-based organizations.
\item Cryptographic protocols do not obviously offer a third system in addition to direct democracy and markets, but techno-anarchists can use cryptography to facilitate these organizational structures.
\item If the techno-anarchist community does decide upon panarchism, then there may not be one kind of organization but several. There may be techno-anarcho-syndicalist unions, or techno-platformist organizations.
\item The only way to determine the correct form of organization will be experience, which depends on techno-anarchists finding each other and attempting to create an organization to see what works.
\item There are already a few people who hold these views, approximately, but they are disorganized and unable to challenge the trends of techno-pessimism and techno-utopianism.
\item Without a techno-anarchist movement, pro-bureaucratic technologies will continue to advance without any resistance from the techno-pessimistic left and the techno-utopian right.
\item It would have been better to take action a long time ago, but the threat may not have been as clear. Now, there is no excuse for the absence of an organized techno-anarchist movement.
\end{enumerate}

\end{document}
